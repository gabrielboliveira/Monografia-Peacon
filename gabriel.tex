%% abtex2-modelo-trabalho-academico.tex, laurocesar
%% Copyright 2012-2015 by abnTeX2 group at http://www.abntex.net.br/ 
%%
%% This work may be distributed and/or modified under the
%% conditions of the LaTeX Project Public License, either version 1.3
%% of this license or (at your option) any later version.
%% The latest version of this license is in
%%   http://www.latex-project.org/lppl.txt
%% and version 1.3 or later is part of all distributions of LaTeX
%% version 2005/12/01 or later.
%%
%% This work has the LPPL maintenance status `maintained'.
%% 
%% The Current Maintainer of this work is the abnTeX2 team, led
%% by Lauro César Araujo. Further information are available on 
%% http://www.abntex.net.br/
%%
%% This work consists of the files abntex2-modelo-trabalho-academico.tex,
%% abntex2-modelo-include-comandos and abntex2-modelo-references.bib
%%

% ------------------------------------------------------------------------
% ------------------------------------------------------------------------
% abnTeX2: Modelo de Trabalho Academico (tese de doutorado, dissertacao de
% mestrado e trabalhos monograficos em geral) em conformidade com 
% ABNT NBR 14724:2011: Informacao e documentacao - Trabalhos academicos -
% Apresentacao
% ------------------------------------------------------------------------
% ------------------------------------------------------------------------

\documentclass[
	% -- opções da classe memoir --
	12pt,				% tamanho da fonte
	openright,			% capítulos começam em pág ímpar (insere página vazia caso preciso)
	oneside,			% para impressão em verso e anverso. Oposto a oneside
	a4paper,			% tamanho do papel. 
	% -- opções da classe abntex2 --
	chapter=TITLE,		% títulos de capítulos convertidos em letras maiúsculas
	%section=TITLE,		% títulos de seções convertidos em letras maiúsculas
	%subsection=TITLE,	% títulos de subseções convertidos em letras maiúsculas
	%subsubsection=TITLE,% títulos de subsubseções convertidos em letras maiúsculas
	% -- opções do pacote babel --
	english,			% idioma adicional para hifenização
	brazil				% o último idioma é o principal do documento
	]{abntex2}


% ----------------------------------------------------------
% Pacotes básicos 
% ----------------------------------------------------------
\usepackage{helvet}			% Usa a fonte Latin Modern - Mudei para Helvetica
\usepackage[T1]{fontenc}		% Selecao de codigos de fonte.
\usepackage[utf8]{inputenc}	% Codificacao do documento (conversão automática dos acentos)
\usepackage{lastpage}		% Usado pela Ficha catalográfica
\usepackage{indentfirst}		% Indenta o primeiro parágrafo de cada seção.
\usepackage{color}			% Controle das cores
\usepackage{graphicx}		% Inclusão de gráficos
\usepackage{microtype} 		% para melhorias de justificação
% ----------------------------------------------------------
		

% ----------------------------------------------------------
% Pacotes adicionais, usados apenas no âmbito do Modelo Canônico do abnteX2
%% ----------------------------------------------------------
\usepackage{lipsum}				% para geração de dummy text
\usepackage{customizacoes} 		% customizações feitas pelo autor
% ----------------------------------------------------------


% ----------------------------------------------------------
% Pacotes de citações
% ----------------------------------------------------------
\usepackage[brazilian,hyperpageref]{backref}	% Paginas com as citações na bibl
\usepackage[alf]{abntex2cite}				% Citações padrão ABNT


% ----------------------------------------------------------
% CONFIGURAÇÕES DE PACOTES
% ----------------------------------------------------------

% ----------------------------------------------------------
% Configurações do pacote backref
% ----------------------------------------------------------

% Usado sem a opção hyperpageref de backref
\renewcommand{\backrefpagesname}{ }
% Texto padrão antes do número das páginas
\renewcommand{\backref}{\ABNTEXchapterfont}
% Define os textos da citação
\renewcommand*{\backrefalt}[4]{
	\ifcase #1 %
		%
	\or
		%
	\else
		%
	\fi}%

% ----------------------------------------------------------


% ----------------------------------------------------------
% Informações de dados para CAPA e FOLHA DE ROSTO
% ----------------------------------------------------------

\titulo{Desenvolvimento de um Protótipo de Rastreador de Beacons utilizando o Raspberry Pi}
\autor{Gabriel Luiz Bastos Oliveira}
\local{Bauru}
\data{2015}
\orientador{Prof. Dr. Eduardo Martins Morgado}
\instituicao{%
  Universidade Estadual Paulista "Júlio de Mesquita Filho"
  \par
  Faculdade de Ciências - Campus Bauru
  \par
  Departamento de Computação
}
\tipotrabalho{Monografia (Trabalho de Conclusão de Curso)}
% O preambulo deve conter o tipo do trabalho, o objetivo, 
% o nome da instituição e a área de concentração 
\preambulo{Relatório Parcial apresentado como requisito da disciplina de Projeto de Implementação de Sistemas I do curso de Ciência da Computação.}

% ----------------------------------------------------------


% ----------------------------------------------------------
% Configurações de aparência do PDF final
% ----------------------------------------------------------

% alterando o aspecto da cor azul
\definecolor{blue}{RGB}{0,0,0}

% informações do PDF
\makeatletter
\hypersetup{
     	%pagebackref=true,
		pdftitle={\@title}, 
		pdfauthor={\@author},
    	pdfsubject={\imprimirpreambulo},
	    pdfcreator={LaTeX with abnTeX2},
		pdfkeywords={beacon}{raspberry pi}{internet das coisas}{abntex2}{trabalho acadêmico}, 
		colorlinks=true,       		% false: boxed links; true: colored links
    	linkcolor=blue,          	% color of internal links
    	citecolor=blue,        		% color of links to bibliography
    	filecolor=magenta,      		% color of file links
		urlcolor=blue,
		bookmarksdepth=4
}
\makeatother
% ----------------------------------------------------------


% ----------------------------------------------------------
% Espaçamentos entre linhas e parágrafos 
% ----------------------------------------------------------

% O tamanho do parágrafo é dado por:
\setlength{\parindent}{1.3cm}

% Controle do espaçamento entre um parágrafo e outro:
\setlength{\parskip}{0.2cm}  % tente também \onelineskip


% ----------------------------------------------------------
% compila o indice
% ----------------------------------------------------------
\makeindex
% ----------------------------------------------------------


% ----------------------------------------------------------
% Configurações de projeto
% ----------------------------------------------------------
\newif\iffinal
\finalfalse % define se é um arquivo final, se for não for retira umas partes. 

\newif\ifrelatorio
\relatoriotrue % define se é um arquivo final, se for não for retira umas partes. 

\newif\ifabstract
\abstractfalse % define se mostra o abstract em inglês ou não.

\newif\ifresumo
\resumotrue % define se mostra o resumo ou não.

% ----------------------------------------------------------


% ----------------------------------------------------------
% Início do documento
% ----------------------------------------------------------
\begin{document}

% Seleciona o idioma do documento (conforme pacotes do babel)
%\selectlanguage{english}
\selectlanguage{brazil}

% Retira espaço extra obsoleto entre as frases.
\frenchspacing 


% ----------------------------------------------------------
% ELEMENTOS PRÉ-TEXTUAIS
% ----------------------------------------------------------
\pretextual

\ABNTEXchapterfont {


% ----------------------------------------------------------
% Capa
% ----------------------------------------------------------
\imprimircapa
% ----------------------------------------------------------


% ----------------------------------------------------------
% Folha de rosto
% (o * indica que haverá a ficha bibliográfica)
% ----------------------------------------------------------
\imprimirfolhaderosto
% ----------------------------------------------------------


% ----------------------------------------------------------
% Inserir a ficha bibliografica
% ----------------------------------------------------------

% Isto é um exemplo de Ficha Catalográfica, ou ``Dados internacionais de
% catalogação-na-publicação''. Você pode utilizar este modelo como referência. 
% Porém, provavelmente a biblioteca da sua universidade lhe fornecerá um PDF
% com a ficha catalográfica definitiva após a defesa do trabalho. Quando estiver
% com o documento, salve-o como PDF no diretório do seu projeto e substitua todo
% o conteúdo de implementação deste arquivo pelo comando abaixo:
%
% \begin{fichacatalografica}
%     \includepdf{fig_ficha_catalografica.pdf}
% \end{fichacatalografica}

\iffinal
  \begin{fichacatalografica}
	\sffamily
	\vspace*{\fill}					% Posição vertical
	\begin{center}					% Minipage Centralizado
	\fbox{\begin{minipage}[c][8cm]{13.5cm}		% Largura
	\small
	\imprimirautor
	%Sobrenome, Nome do autor
	
	\hspace{0.5cm} \imprimirtitulo  / \imprimirautor. --
	\imprimirlocal, \imprimirdata-
	
	\hspace{0.5cm} \pageref{LastPage} p. : il. (algumas color.) ; 30 cm.\\
	
	\hspace{0.5cm} \imprimirorientadorRotulo~\imprimirorientador\\
	
	\hspace{0.5cm}
	\parbox[t]{\textwidth}{\imprimirtipotrabalho~--~\\ \imprimirinstituicao,
	\imprimirdata.}\\
	
	\hspace{0.5cm}
		1. Beacon.
		2. Raspberry Pi.
		3. Internet das Coisas.
		I. \imprimirorientador.
		II. Universidade Estadual Paulista "Júlio de Mesquita Filho".
		III. Faculdade de Ciências.
		IV. Título
	\end{minipage}}
	\end{center}
  \end{fichacatalografica}
\fi
% ----------------------------------------------------------


% ----------------------------------------------------------
% Inserir errata
% ----------------------------------------------------------
%\begin{errata}
%Elemento opcional da \citeonline[4.2.1.2]{NBR14724:2011}. Exemplo:

%\vspace{\onelineskip}

%FERRIGNO, C. R. A. \textbf{Tratamento de neoplasias ósseas apendiculares com
%reimplantação de enxerto ósseo autólogo autoclavado associado ao plasma
%rico em plaquetas}: estudo crítico na cirurgia de preservação de membro em
%cães. 2011. 128 f. Tese (Livre-Docência) - Faculdade de Medicina Veterinária e
%Zootecnia, Universidade de São Paulo, São Paulo, 2011.

%\begin{table}[htb]
%\center
%\footnotesize
%\begin{tabular}{|p{1.4cm}|p{1cm}|p{3cm}|p{3cm}|}
%  \hline
%   \textbf{Folha} & \textbf{Linha}  & \textbf{Onde se lê}  & \textbf{Leia-se}  \\
%    \hline
%    1 & 10 & auto-conclavo & autoconclavo\\
%   \hline
%\end{tabular}
%\end{table}

%\end{errata}
% ----------------------------------------------------------


% ----------------------------------------------------------
% Inserir folha de aprovação
% ----------------------------------------------------------

% Isto é um exemplo de Folha de aprovação, elemento obrigatório da NBR
% 14724/2011 (seção 4.2.1.3). Você pode utilizar este modelo até a aprovação
% do trabalho. Após isso, substitua todo o conteúdo deste arquivo por uma
% imagem da página assinada pela banca com o comando abaixo:
%
% \includepdf{folhadeaprovacao_final.pdf}
%
%\begin{folhadeaprovacao}
%  \ABNTEXchapterfont {

%    \begin{center}
    
%      {\ImprimirAutor}

%      \vspace*{\fill}\vspace*{\fill}
      
%      \begin{center}
%        \bfseries\large\ImprimirTitulo
%      \end{center}
      
%      \vspace*{\fill}
    
%      \hspace{.45\textwidth}
%      \begin{minipage}{.5\textwidth}
%          \imprimirpreambulo
%      \end{minipage}%
%      \vspace*{\fill}
%     \end{center}
        
     %Trabalho aprovado. \imprimirlocal, 24 de novembro de 2012:

%     \assinatura{\textbf{\imprimirorientador} \\ Orientador} 
     %\assinatura{\textbf{Professor} \\ Convidado 1}
     %\assinatura{\textbf{Professor} \\ Convidado 2}
     %\assinatura{\textbf{Professor} \\ Convidado 3}
     %\assinatura{\textbf{Professor} \\ Convidado 4}
%      \vspace*{0.5cm}
%      \hspace{.5\textwidth}
%     \begin{center} 
%       \ImprimirLocal \\ \imprimirdata
%     \end{center}
%  }
%\end{folhadeaprovacao}
% ----------------------------------------------------------


% ----------------------------------------------------------
% Dedicatória
% ----------------------------------------------------------
\iffinal
  \begin{dedicatoria} 
   \vspace*{\fill}
   \centering
   \noindent
   \textit{ Este trabalho é dedicado às crianças adultas que,\\
   quando pequenas, sonharam em se tornar cientistas.} \vspace*{\fill}
  \end{dedicatoria}
\fi
% ----------------------------------------------------------


% ----------------------------------------------------------
% Agradecimentos
% ----------------------------------------------------------
\iffinal
  \begin{agradecimentos}
Os agradecimentos principais são direcionados à Gerald Weber, Miguel Frasson,
Leslie H. Watter, Bruno Parente Lima, Flávio de Vasconcellos Corrêa, Otavio Real
Salvador, Renato Machnievscz\footnote{Os nomes dos integrantes do primeiro
projeto abn\TeX\ foram extraídos de
\url{http://codigolivre.org.br/projects/abntex/}} e todos aqueles que
contribuíram para que a produção de trabalhos acadêmicos conforme
as normas ABNT com \LaTeX\ fosse possível.

Agradecimentos especiais são direcionados ao Centro de Pesquisa em Arquitetura
da Informação\footnote{\url{http://www.cpai.unb.br/}} da Universidade de
Brasília (CPAI), ao grupo de usuários
\emph{latex-br}\footnote{\url{http://groups.google.com/group/latex-br}} e aos
novos voluntários do grupo
\emph{\abnTeX}\footnote{\url{http://groups.google.com/group/abntex2} e
\url{http://www.abntex.net.br/}}~que contribuíram e que ainda
contribuirão para a evolução do \abnTeX.

\end{agradecimentos}
\fi
% ----------------------------------------------------------


% ----------------------------------------------------------
% Epígrafe
% ----------------------------------------------------------
\iffinal
  \begin{epigrafe}
    \vspace*{\fill}
	\begin{flushright}
		\textit{``Não vos amoldeis às estruturas deste mundo, \\
		mas transformai-vos pela renovação da mente, \\
		a fim de distinguir qual é a vontade de Deus: \\
		o que é bom, o que Lhe é agradável, o que é perfeito.\\
		(Bíblia Sagrada, Romanos 12, 2)}
	\end{flushright}
  \end{epigrafe}
\fi
% ----------------------------------------------------------


% ----------------------------------------------------------
% RESUMOS
% ----------------------------------------------------------
\ifresumo
	% resumo em português
	\setlength{\absparsep}{18pt} % ajusta o espaçamento dos parágrafos do resumo
	\begin{resumo}
		\ABNTEXchapterfont {
	 		Atualmente a área da internet das coisas tem recebido um foco enorme. Um dos motivos é devido a ser bem nova e também facilitar a integração das pessoas com o mundo real. Essa área tem motivado várias pessoas pela possibilidade de ter um produto concreto após pouco tempo de trabalho. Por conta disso, alguns dispositivos foram criados para auxiliar essas áreas, entre eles o \textit{Arduino} e \textit{Raspberry Pi}. Os \textit{beacons} também são uma criação nova, comunicando com outros dispositivos maiores como o \textit{RPi} para auxiliar a micro-localização dentro de um pequeno espaço, com um baixo custo. O seu foco de uso é pequeno, somente com leituras via \textit{smartphones}. Esse projeto tem como objetivo criar um rastreador de \textit{beacons} utilizando o \textit{Raspberry Pi} para expandir essa capacidade, não necessitando de um celular para que os \textit{beacons} possam ser encontrados.

 			\textbf{Palavras-chave}: \textit{Beacon}. Raspberry Pi. Internet das Coisas.
		}
	\end{resumo}

	\ifabstract
		% resumo em inglês
		\begin{resumo}[Abstract]
	 		\begin{otherlanguage*}{english}
				This is the english abstract.

				\vspace{\onelineskip}
 
				\noindent 
				\textbf{Keywords}: Beacon. Raspberry Pi. Internet of Things.
			\end{otherlanguage*}
		\end{resumo}
	\fi
\fi
% ----------------------------------------------------------


% ----------------------------------------------------------
% inserir lista de ilustrações
% ----------------------------------------------------------
\iffinal
	\pdfbookmark[0]{\listfigurename}{lof}
	\ABNTEXchapterfont {		
		\listoffigures*
	}
	\cleardoublepage
\fi
% ----------------------------------------------------------


% ----------------------------------------------------------
% inserir lista de tabelas
% ----------------------------------------------------------
\iffinal
	\ABNTEXchapterfont {
		\pdfbookmark[0]{\listtablename}{lot}
		\listoftables*
		\cleardoublepage
	}
\fi
% ----------------------------------------------------------


% ----------------------------------------------------------
% inserir lista de abreviaturas e siglas
% ----------------------------------------------------------
\iffinal
	\begin{siglas}
			\item[IoT] \textit{Internet of Things} - Internet das Coisas
			\item[DIY] \textit{Do It Yourself} - Faça Você Mesmo
			\item[RPi] \textit{Raspberry Pi}
			\item[GPIO] \textit{General Input and Output}
			\item[BLE] \textit{Bluetooth Low Energy}
			\item[LTIA] Laboratório de Tecnologia da Informação Aplicada
	\end{siglas}
\fi
% ----------------------------------------------------------


% ----------------------------------------------------------
% inserir lista de símbolos
% ----------------------------------------------------------
\iffinal
	\begin{simbolos}
		\ABNTEXchapterfont {
			\item[$ \Gamma $] Letra grega Gama
			\item[$ \Lambda $] Lambda
			\item[$ \zeta $] Letra grega minúscula zeta
			\item[$ \in $] Pertence
		}
	\end{simbolos}
\fi
% ----------------------------------------------------------


% ----------------------------------------------------------
% inserir o sumario
% ----------------------------------------------------------
\pdfbookmark[0]{\contentsname}{toc}
\tableofcontents*
\cleardoublepage
% ----------------------------------------------------------



% ----------------------------------------------------------------------------------------------------------------------------------



% ----------------------------------------------------------------------------------------------------------------------------------
% ELEMENTOS TEXTUAIS
% ----------------------------------------------------------------------------------------------------------------------------------
\textual


% ----------------------------------------------------------
% Introdução
% ----------------------------------------------------------
\chapter[Introdução]{Introdução}
%\addcontentsline{toc}{chapter}{Introdução}
% ----------------------------------------------------------

A internet cresce dia após dia e cada vez mais diferentes tipos de dispositivos são conectados a essa imensa rede. Há uma estimativa de 26 bilhões de ''coisas'' conectadas a internet até 2020, comparado a 6 bilhões na década de 2000. Isso foi possível graças ao custo do acesso a internet e largura de banda (quantidade de dados trafegados) ter diminuído 40 vezes nos últimos 10 anos. Esse \textit{boom} de crescimento também é influenciado pela IoT (\textit{Internet of Things} - Internet das Coisas). \cite{goldmansachs-iot}. 

Segundo \citeonline{ashton-iot} o termo IoT surgiu como título de sua apresentação a Procter \& Gamble (P\&G) em 1999. Esse termo se dá ao uso de internet em diferentes tipos de dispositivos. 

\begin{citacao}
''(...) a Internet das Coisas é um conceito no qual dispositivos de nosso dia a dia são equipados com sensores capazes de captar aspectos do mundo real, como por exemplo, temperatura, umidade, presença, etc, e envia-los a centrais que recebem estas informações e as utilizam de forma inteligente.'' \cite{nascimento-iot}.
\end{citacao}

Como exemplo podemos citar: geladeira ligada a internet informando a falta de condimentos, caixa de remédio conectada a internet prevendo o término da caixa de remédio para avisar ao consumidor, entre diversos outros. Um bom exemplo é interligar uma casa por meio de sensores, como termostatos, sistemas de segurança, iluminação, sistemas de entretenimento com uma inteligência por trás para diversas aplicações. \cite{goldmansachs-iot}

Diversas áreas tiveram o seu crescimento alavancado por conta da IoT. O mais marcante é o ramo do DIY (\textit{Do It Yourself}, ou faça você mesmo), em que pessoas criam ou adaptam coisas para suas necessidades. Isso se dá por conta de ter aparecido no mercado dispositivos e componentes que auxiliam a criação e adaptação de eletrônicos e outros. Um ótimo exemplo é o Arduino, um dispositivo que nos ajuda a criar os projetos de eletrônica que consiste de duas partes: o hardware e o software. Com eles é possível construir praticamente de tudo, desde um LED piscante a um robô que envia um \textit{tweet} quando sua planta está sem água. \cite{ben-arduino}. \cite{sorrel-arduino}.

Além do Arduino existem diversos outros \textit{devices} com funcionalidades similares ou até complementares. Podemos citar o \textit{Raspberry Pi}, um computador do tamanho de um cartão de crédito e de baixo custo. \cite{raspberrypi-rpi}. Por ser um computador é possível executar um sistema operacional (como Linux, \textit{RISC OS}, \textit{Windows 10 for IoT09}). Dependendo do modelo possui portas USB (1 a 4) para conexão com periféricos (como mouse, teclado, adaptador WiFi, Bluetooth), e também porta Ethernet para conexão a internet cabeada (exceto modelo A e A+). Também possui de 26 (modelo A e B) a 40 pinos (modelo A+, B+ e 2) para conexões gerais de entrada e saída digitais (GPIO - \textit{General Input and Output}). 

Através desses pinos pode-se conectar uma diversidade de componentes eletrônicos como sensores, atuadores, outros dispositivos para comunicação. Dessa forma, suas funcionalidades são expandidas de uma forma absurda, ficando a cargo de cada pessoa montar uma nova aplicação. Uma boa aplicação é a conexão de um adaptador bluetooth pela USB para comunicação com smartphones, tablets, PCs e outros dispositivos tipo sensores sem fio.

Um exemplo desses sensores externos são os \textit{beacons}, pequenos modelos sem fio que se comunicam por meio do Bluetooth 4.0 ou BLE (\textit{Bluetooth Low Energy} - Bluetooth de Baixa Energia). Como o próprio nome diz, esse padrão de comunicação utiliza muita pouca energia. Desta forma, um \textit{beacon} pode funcionar por anos. ''Na prática, ela permite localizar objetos (ou pessoas que carregam esses objetos) com muito mais precisão dentro de ambientes fechados.'' \cite{teixeira-beacon}.

% ----------------------------------------------------------


% ----------------------------------------------------------
% Objetivos
% ----------------------------------------------------------
\chapter{Objetivos}
% ----------------------------------------------------------

% ---
% Objetivos Gerais
% ---
\section{Objetivos Gerais}
% ---

O objetivo geral desse projeto é planejar e desenvolver um protótipo de rastreador de \textit{iBeacons} utilizando o \textit{Raspberry Pi}.

% ---

% ---
\section{Objetivos Específicos}
% ---

\begin{alineas}
	\item Estudar o funcionamento de \textit{beacons}, comunicação via \textit{bluetooth low energy} e \textit{Raspberry Pi}, assim como suas aplicações.
	\item Identificar os requerimentos básicos de funcionamento de \textit{beacons} e \textit{Raspberry Pi}.
	\item Definir os elementos para desenvolvimento do protótipo.
	\item Planejar a estrutura do sistema.
	\item Implementar o protótipo de acordo com a estrutura e elementos planejados.
\end{alineas}

% ----------------------------------------------------------


% ----------------------------------------------------------
% Fundamentação Teórica
% ----------------------------------------------------------
\chapter{Fundamentação Teórica}
% ----------------------------------------------------------

A área de IoT está em uma onda crescente, com um grande número de pessoas gastando nesse mercado, e um bom número de profissionais migrando para essa área. Há uma estimativa de que existem 19 milhões de profissionais trabalhando na indústria de desenvolvimento de software, e desses, 19\% trabalham em algum projeto relacionado a IoT. \cite{cw-iot}.

\begin{citacao}
''A próxima onda na era da computação será fora do domínio do ambiente de trabalho tradicional. No paradigma da IoT, muitos dos objetos que nos rodeiam estarão na rede de uma forma ou de outra. RFID (Radio Frequency Identification - Identificadores via Rádio Frequência) e as tecnologias de redes de sensores crescerão para enfrentar este novo desafio, em que os sistemas de informação e comunicação estão embutidos nos ambientes que nos rodeiam, de forma invisível.''. \cite{iot-article}. 
\end{citacao}

É notável o crescimento dessa área. Há uma expectativa de que, em 2020, o número de carros conectados a internet supere o número de carros não conectados, sendo esses carros possíveis de se comunicar com outros veículos e a infraestrutura das ruas, como os semáforos. \cite{goldmansachs-iot}. Segundo \citeonline{press-iot}, em 2014 a IoT substituiu a área da \textit{Big Data} como a tecnologia mais empolgante, ou seja, a tecnologia que mais pessoas iriam migrar e se interessar. 


% ---
\section{Raspberry Pi}
% ---

Quando se pesquisa algo relacionado a IoT é praticamente impossível não achar relação e links para \textit{Raspberry Pi}, Arduinos, e toda a área de \textit{DIY}. Atualmente existem três modelos, conforme a tabela~\ref{table:comparativo-rpi} e figura~\ref{fig:todos-rpi}. 

\begin{figure}[h!]
	\ABNTEXchapterfont {
		\centering
		\includegraphics[width=1\textwidth]{img/rpi-modelos.jpg}
		\caption{\textit{RPi} versão A+ (esquerda), versão B+ (centro) e versão 2 Modelo B (direita). Elaborado pelo autor.}
		\label{fig:todos-rpi}
	}
\end{figure}


\begin{table}[htb]
	\IBGEtab{%
		\ABNTEXchapterfont {
  			\caption{Comparativo entre os modelos de \textit{Raspberry Pi}}%
  			\label{table:comparativo-rpi}
		}
	}{%
  		\begin{tabular}{cccc}
	  		\toprule
	 			Versão & A+ & B+ & 2 Modelo B \\
  			\midrule \midrule
				Processador & ARMv6 single core & ARMv6 single core & ARMv7 quad core \\
			\midrule
				Velocidade CPU & 700 MHz single-core & 700 MHz single-core & 900 MHz quad-core \\
			\midrule 
				Memória RAM & 256 MB & 512 MB & 1 GB \\
  			\midrule 
				Portas USB & 1  & 4  & 4 \\
			\midrule 
				\textit{Ethernet} & Não & Sim  & Sim \\
  			\bottomrule
		\end{tabular}%
	}{%
  		\fonte{\cite{rpiversion-table}}%
  	}
\end{table}

Segundo \citeonline{raspberrypi-rpi}, a fundação responsável pela criação desse pequeno dispositivo gostaria de ver ele ser usado por crianças e pessoas carentes do mundo todo para aprender a programar e como a computação funciona. É facilmente configurável, utilizando um cartão de memória como HD para salvar dados e o sistema operacional.

Por meio de suas portas de entrada e saída digitais é possível conectar uma gama ampla de sensores, atuadores, compontentes eletrônicos, ficando a cargo do programador e criador do projeto a escolher como esses pinos serão conectados e aproveitados. Atualmente existem diversos módulos para expandir os meios de comunicação entre diferentes \textit{devices}, e um deles é via \textit{bluetooth}.

Atualmente existem diferentes sistemas operacionais portados para o \textit{RPi}. Entre eles, temos os seguintes:

\begin{alineas}
	\item \textbf{Raspbian}: baseado na distribuição \textit{Linux} chamada \textit{Debian}. Atualmente é a suportada oficialmente pela \textit{RPi Foundation}. \cite{rpi-download}.
	\item \textbf{Ubuntu Mate}: baseado na distribuição \textit{Linux} chamada \textit{Ubuntu}, juntamente com o software \textit{MATE Desktop} para gerenciamento de janelas. \cite{ubuntu-mate}.
	\item \textbf{Snappy Ubuntu Core}: distribuição \textit{Linux} voltada a \textit{cloud} e dispositivos. \cite{snappy-ubuntu}.
	\item \textbf{Windows 10 for IOT Core}: versão do \textit{Windows} 10 da \textit{Microsoft} para dispositivos voltados a internet das coisas. \cite{windows10-iot}.
\end{alineas}

% ---
\section{Bluetooth Low Energy}
% ---

A tecnologia do \textit{bluetooth} vem sendo amplamente utilizada, principalmente após a criação da versão 4.0, ou \textit{BLE} (\textit{Bluetooth Low Energy} - Bluetooth de Baixa Energia), por conta de ter um baixíssimo gasto energético, podendo preservar a bateria do dispositivo que a utiliza. O \textit{BLE} faz parte da tecnologia nomeada \textit{Bluetooth Smart} inteligente e eficiente energeticamente, voltada para \textit{devices} que usam pequenas fontes de energia. \cite{bluetooth-smart}.

O \textit{BLE} possui similaridades com a versão clássica do \textit{bluetooth}. Ambos utilizam o espectro de frequências de 2.4 GHz (mesmo utilizado pelas redes \textit{WiFi}), mesma modulação GFSK e velocidade de 1 Mbps, porém a indexação de ambos é diferente. A versão clássica possui 79 canais, e a \textit{BLE} possui 40 canais. Além disso, os canais são espaçados de forma diferente, conforme tabela~\ref{table:ble-physical}. \cite{ble-packets}.

\begin{table}[htb]
	\IBGEtab{%
		\ABNTEXchapterfont {
  			\caption{\textit{BLE Physical Layer}}%
  			\label{table:ble-physical}
		}
	}{%
  		\begin{tabular}{ccc}
	  		\toprule
	 			  & BLE & Classic \\
  			\midrule \midrule
				Modulação & GFSK 0.45 a 0.55 & GFSK 0.28 a 0.35 \\
			\midrule
				Velocidade de Transferência & 1 Mbit/s & 1 Mbit/s \\
			\midrule 
				Canais & 40 & 79 \\
  			\midrule 
				Espaçamento & 2 MHz & 1 MHz \\
  			\bottomrule
		\end{tabular}%
	}{%
  		\fonte{\cite{ble-packets}}%
  	}
\end{table}

O espectro de 2.4 GHz para \textit{bluetooth} se extende de 2402 MHz a 2480 MHz, e os canais 37, 38 e 39 (últimos três) são específicos para anúncio (\textit{advertisement}), conforme figura~\ref{fig:banda-channels}.  \cite{ble-packets}. 

\begin{figure}[h!]
	\ABNTEXchapterfont {
		\centering
		\includegraphics[width=0.8\textwidth]{img/banda-2-4.png}
		\caption{Separação da banda 2.4 GHz para bluetooth e WiFi. \cite{ble-packets}.}
		\label{fig:banda-channels}
	}
\end{figure}

Interessante notar que esses canais estão posicionados em forma bastante estratégica, no começo, final e meio da banda de 2.4 GHz. Isso se deve para aumentar a eficácia, evitando todos os canais ficarem lotados ou com muita interferência. \cite{ble-packets}. 

Os \textit{BLE Advertisement Packets}, ou pacotes de anúncio \textit{BLE} é uma das formas de conexão do \textit{Bluetooth Smart}. Por meio dos anúncios, um \textit{device} transmite pacotes para todos que estão a sua volta, sem necessariamente necessitar de uma conexão direta entre somente outro dispositivo.

Um \textit{BLE Advertisement Packet} é formado conforme figura~\ref{fig:ble-adv-packet}. O preâmbulo, \textit{access address} e CRC são informações para formação do pacote. Os dados estão de fato dentro do PDU (\textit{Packet Data Unit}). O cabeçalho de 2 bytes informa o tamanho do \textit{payload} (carga de dados), além de informações relevantes como tipo do pacote, tipo de mensagem enviada, entre outros. \cite{ble-packets}.

\begin{figure}[h!]
	\ABNTEXchapterfont {
		\centering
		\includegraphics[width=1\textwidth]{img/ble-adv-packet.png}
		\caption{Modelo de \textit{BLE Advertising Packet}. \cite{ble-packets}.}
		\label{fig:ble-adv-packet}
	}
\end{figure}

O importante do \textit{BLE Advertising Packet} é o tipo de anúncio feito, ou quais são as informações do pacote. \citeonline{gap-ble} apresenta uma tabela com os possíveis valores e também o significado de cada uma. Por exemplo, o valor 0xFF significa que o pacote contém dados específicos do fabricante, ou seja, existe a flexibilidade de manipular o pacote da forma que for precisa, contanto que mantenha a estrutura original de 6 a 37 bytes de \textit{payload}, conforme figura~\ref{fig:ble-adv-packet}. \cite{ble-packets}.

% ---
\section{Beacon}
% ---

Os \textit{beacons} são pequenos sensores que são capazes de identificar objetos com precisão dentro de ambientes fechados. \cite{teixeira-beacon}.

\begin{citacao}
Como muitos espaços fechados (restaurantes, museus, shopping centers, casas de show) possuem estrutura metálica ou utilizam algum tipo de metal em sua construção, é comum que o sinal de GPS fique enfraquecido quando os usuários estão dentro daquele local. Nesse caso, os \textit{Beacons} são uma ótima solução: um hardware relativamente barato, e pequeno o suficiente para ser plugado na parede ou instalado sobre um balcão. \cite{teixeira-beacon}.
\end{citacao}

\begin{figure}[h!]
	\ABNTEXchapterfont {
		\centering
		\includegraphics[width=0.6\textwidth]{img/beacon-mpact.jpg}
		\caption{Modelo de \textit{beacon} proprietário - MPact, da Zebra Technologies Corporation. Elaborado pelo autor.}
		\label{fig:beacon-mpact}
	}
\end{figure}

Segundo \citeonline{teixeira-beacon}, os \textit{beacons} utilizam o \textit{BLE} para detectar um dispositivo próximo e transmitir seu identificador único e avisar que está ali presente. \citeonline{teixeira-beacon} também diz que os \textit{beacons} não são inteligentes, toda a interação deve depender do dispositivo que recebe a informação do identificador único.

Atualmente os usos de \textit{beacons} estão restritos a aplicativos em smartphones realizando a leitura e interagindo com o usuário, porém existem ainda diversas áreas a serem exploradas, e um bom exemplo são as casas inteligentes. Segundo \citeonline{grothaus-smarthomes}, a \textit{Apple} está apostando em um kit de desenvolvimento (\textit{HomeKit}) que permita aos desenvolvedores interagirem com \textit{smart devices} presentes no ambiente.

O protocolo \textit{iBeacon} foi apresentado pela Apple juntamente com o iOS 7, versão de seu sistema operacional para dispositivos móveis. É uma tecnologia baseada nos \textit{beacons}, porém adaptada para as necessidades e aplicações de seu sistema móvel.

Apple criou uma adaptação do pacote genérico de \textit{beacon} para transmitir um total de três dados:

\begin{alineas}
	\item \textbf{UUID}: Identificador único formado de 16 bytes (128 bits). Focado em ser único para cada aplicação, cada aplicativo ou desenvolvimento deve ter o seu.
	\item \textbf{Major}: Identificador de 2 bytes que identifica uma sub-região grande. Usado, por exemplo, para dividir as lojas de um grande varejista.
	\item \textbf{Minor}: Identificador de 2 bytes que identifica uma sub divisão de região, ou seja, uma região menor que o Minor.
\end{alineas}

Um exemplo de aplicação é citado na tabela~\ref{table:stores-apple}. Utiliza-se um único UUID para todas as lojas, com um único Major por loja e um Minor por departamento, podendo ser repetido entre as lojas. Dessa forma amplia-se o número de possíveis aplicações, aumentando a quantidade de números disponíveis.

\begin{table}[htb]
	\IBGEtab{%
		\ABNTEXchapterfont {
  			\caption{\textit{Exemplo de aplicação}}%
  			\label{table:stores-apple}
		}
	}{%
  		\begin{tabular}{cccc}
	  		\toprule
	 			Localização da Loja & São Francisco & Paris & Londres \\
  			\midrule \midrule
				UUID & \multicolumn{3}{c}{D9B9EC1F-3925-43D0-80A9-1E39D4CEA95C} \\
			\midrule
				Major & 1 & 2 & 3 \\
			\midrule 
				Minor - Roupas & 10 & 10 & 10 \\
  			\midrule 
				Minor - Utilidades Domésticas & 20 & 20 & 20 \\
			\midrule 
				Minor - Automotivo & 30 & 30 & 30 \\
  			\bottomrule
		\end{tabular}%
	}{%
  		\fonte{\cite{ibeacon-apple}}%
  	}
\end{table}

O pacote \textit{BLE} utilizado pela tecnologia iBeacon pode ser visto na figura~\ref{fig:ibeacon-packet}. Segundo \citeonline{arm-beacons}, o significado do prefixo iBeacon é:

\begin{alineas}
	\item \textit{\textbf{Adv Flags}}: determinam que é um pacote \textit{BLE} de descobrimento geral, e que somente transmite e não permite conexões.
	\item \textit{\textbf{Adv Header}}: determinam que os próximos 26 bytes serão a carga de dados (\textit{payload} de fato). Sempre será 0x1AFF.
	\item \textit{\textbf{Company ID}}: indica que é o ID da Apple junto com a Bluetooth SIG. Essa informação que faz ser dependente da Apple. Sempre será 0x004C.
	\item\textit{\textbf{iBeacon Type}}: ID secundário utilizado por todos iBeacons que identificam ser um \textit{beacon} de proximidade. Sempre será 0x02.
	\item\textit{\textbf{iBeacon Length}}: identifica quantos bytes terão em seguida. Sempre será 0x15, ou 21 bytes.
\end{alineas}

\begin{figure}[h!]
	\ABNTEXchapterfont {
		\centering
		\includegraphics[width=0.9\textwidth]{img/ibeacon-packet.png}
		\caption{\textit{Payload} do pacote iBeacon. \cite{arm-beacons}.}
		\label{fig:ibeacon-packet}
	}
\end{figure}

Os \textit{iBeacons} foram criados com intuito de serem descobertos por smartphones. Um exemplo de aplicação é uma cafeteria com um \textit{iBeacon} no balcão próximo ao caixa. Quando um consumidor entra na loja e chega próximo ao caixa, um aplicativo em seu celular identifica o \textit{iBeacon} pela sua UUID, identifica pelo Major que se trata da cafeteria número 12 e encontra uma promoção com o Minor de número 26. Em seguida, apresenta uma notificação ao usuário uma promoção e também um cupom válido de desconto para usar no caixa. \cite{arm-beacons}.

Uma outra aplicação interessante citada por \citeonline{arm-beacons} é a possibilidade do smartphone transmitir pacotes \textit{iBeacon}, sem necessidade de um hardware externo. Dessa forma, pode-se por exemplo automatizar o check-in em um evento e rastrear o movimento dos usuários entre os estabelecimentos.

Existem outros protocolos baseados na tecnologia \textit{beacon}. Dois se destacam por ser abertos e passíveis de alterações: \textit{AltBeacon} e \textit{Eddystone}, este último criado pela Google. O \textit{AltBeacon} possui um modelo bastante parecido com o \textit{iBeacon}, porém com possibilidade de alterar o número do fabricante, ter possibilidade de código de \textit{beacons} diferentes, e também a possibilidade do fabricante colocar sua informação ao final do pacote, conforme figura~\ref{fig:altbeacon-packet} \cite{arm-beacons}.

\begin{figure}[h!]
	\ABNTEXchapterfont {
		\centering
		\includegraphics[width=1\textwidth]{img/altbeacon-packet.png}
		\caption{Pacote do \textit{AltBeacon}. \cite{arm-beacons}.}
		\label{fig:altbeacon-packet}
	}
\end{figure}

Segundo \citeonline{eddystone-google}, o protocolo \textit{Eddystone} possui três modos de funcionamento:
\begin{alineas}
	\item \textit{Eddystone}-UID: transmite um \textit{beacon} ID, composto de 10 bytes identificando um grupo de \textit{beacons} e  6 bytes identificando um único \textit{beacon}.
	\item \textit{Eddystone}-URL: transmite um link comprimido, para que o cliente possa acessar um site na internet.
	\item \textit{Eddystone}-TLM: transmite informações de telemetria sobre o \textit{beacon}, como por exemplo voltagem da bateria, temperatura e quantos pacotes foram enviados.
\end{alineas}

% ----------------------------------------------------------


% ----------------------------------------------------------
% Proposta do Projeto
% ----------------------------------------------------------
\chapter{Proposta do Projeto}
% ----------------------------------------------------------

Os \textit{beacons} foram pouco explorados até o momento, seu uso está sendo mais notado na área de grandes lojas do varejo.

\begin{citacao}
''A Apple (...) já está utilizando a tecnologia em 254 lojas nos EUA. As funcionalidades já estão embutidas na versão oficial do aplicativo da Apple Store para iOS [, sistema operacional de seus smartphones e tablets]. (...) Quando o usuário se aproxima de uma loja física, o aplicativo oferece toda uma camada extra de informações e serviços que são específicos para aquela unidade – como por exemplo ofertas locais, tamanho da fila para ser atendido no Genius Bar, eventos e treinamentos que estão agendados ali na loja etc.'' \cite{teixeira-beacon}.
\end{citacao}

A Macy's (grande rede norte americana de loja de departamentos) está realizando testes em algumas de suas lojas para enviar alertas a pessoas que entrarem em suas lojas, com promoções e melhores indicações, utilizando o padrão \textit{iBeacon} da Apple. Até o momento esse teste está limitado a usuários de iPhone, e somente quando entrar na loja. Em um teste futuro espera-se que seja possível separar por departamentos, para que quando um usuário percorra a loja apareça as notícias relativo ao local da loja que ela está. \cite{kastrenakes-macys-beacon}

Porém as maiores aplicações são voltadas a smartphones e tablets como receptores e identificadores dos \textit{beacons}. Existem diversas empresas desenvolvendo variados tipos de \textit{beacons}, mas a maioria delas utiliza aplicativos para \textit{mobile devices} como identificadores. 

Alguns motivos podem ser apresentados para isso, mas o mais relevante é que até o momento poucas pessoas, grupos e empresas dedicaram tempo e dinheiro para investir nessa nova tecnologia. Um grande problema também é a necessidade de ter o \textit{bluetooth} ligado e permitindo conexões para que um smartphone consiga identificar os \textit{beacons}.

Como visto anteriormente, o \textit{RPi} é um computador pequeno e de baixo custo, e também tem portas de entrada e saída digitais e portas USB. Por conta disso é possível expandir suas funcionalidades conectando módulos de \textit{bluetooth}, \textit{WiFi}, entre outros, podendo ser utilizado como um ótimo dispositivo de prototipagem.

Por ter uma comunidade grande e ativa, o seu estudo é facilitado. Um bom número de sites e comunidades estão desenvolvendo projetos com esse dispositivo e publicando suas conquistas. É possível encontrar tutoriais e passo-a-passo de um bom número de projetos, porém somente para aprendizado. Ao desenvolver um projeto mais elaborado isso pode servir como guia de estudo e aprofundamento.

As empresas tem investido um bom dinheiro na área de \textit{DIY} e \textit{IoT}. Existe uma grande expectativa para o futuro, por ser uma tecnologia barata para ser aplicada a grandes volumes, ter bastante gente envolvida e engajada, e também sua grande facilidade de criação.

Esse projeto se propõe a desenvolver um protótipo utilizando outro meio para rastrear os \textit{iBeacons}, neste caso o \textit{RPi}. Desta forma será possível explorar maiores aplicações aos \textit{beacons}.

% ----------------------------------------------------------


% ----------------------------------------------------------
% Método de Pesquisa
% ----------------------------------------------------------
\chapter{Método de Pesquisa}
% ----------------------------------------------------------

O levantamento bibliográfico relacionado ao tema foi a primeira fase dessa pesquisa, foram realizadas buscas relacionadas aos assuntos: \textit{Raspberry Pi}, comunicação via \textit{Bluetooth Low Energy}, \textit{beacons}. Concomitantemente foi realizado o estudo das tecnologias, suas capacidades, limitações, aplicações, etc.

O segundo passo foi planejar o projeto baseado na análise do levantamento bibliográfico, assim como a definição de sua estrutura. Essa etapa foi necessária para facilitar e agilizar a implementação e testes dos componentes no próximo passo, definindo assim um escopo inicial de funcionalidades que o sistema terá, assim como outras tarefas a serem realizadas. Todo o processo será citado no tópico 6 (Experimentos e Resultados).

Em seguida será iniciada a implementação e testes do protótipo. O projeto será desenvolvido por etapas, sendo que a primeira será o estudo de identificação dos \textit{beacons} via smartphone, a segunda será o desenvolvimento do software de identificação no \textit{RPi}, e a terceira um aprimoramento do software embarcado para rastrear os \textit{beacons}. Em seguida serão realizados testes unitários experimentais em laboratório para que o protótipo possa ser aprimorado. Será testado com mais de um modelo de \textit{RPi} e também mais de um \textit{beacon}, em ambiente aberto e fechado. Ao final espera-se ter um protótipo finalizado.

A etapa de testes será realizada por observação e experimentação nos ambientes. Os dados serão apresentados ao final do projeto pela confecção e apresentação da monografia.

Será desenvolvido no espaço do LTIA (Laboratório de Tecnologia da Informação Aplicada), da Universidade Estadual Paulista "Júlio de Mesquita Filho" - Campus Bauru. Os \textit{beacons MPact} e iPad Mini são de propriedade do próprio laboratório.

% ----------------------------------------------------------


% ----------------------------------------------------------
% Materiais
% ----------------------------------------------------------
\chapter{Materiais e Instrumentos}
% ----------------------------------------------------------

A versão escolhida para início do projeto foi a 2 Modelo B, por ter um melhor processamento e mais memória RAM (figura~\ref{fig:rpi2-utilizado}), conforme informado no item 3.2. Para os testes futuros será utilizado também a versão B+.

\begin{figure}[h!]
	\ABNTEXchapterfont {
		\centering
		\includegraphics[width=0.7\textwidth]{img/rpi2.jpg}
		\caption{\textit{RPi} 2 modelo B utilizado nesse projeto. Elaborado pelo autor.}
		\label{fig:rpi2-utilizado}
	}
\end{figure}

É necessário o uso de um adaptador \textit{WiFi} para conexão a internet sem necessidade de cabo \textit{Ethernet} e também um adaptador Bluetooth 4.0 com suporte a \textit{BLE} para fazer a busca dos pacotes \textit{beacon}. Os modelos de adaptadores usados serão Orico BTA-406 (bluetooth) e EDUP N8508GS (\textit{WiFi}), conforme figura~\ref{fig:adaptadores}.

\begin{figure}[h!]
	\ABNTEXchapterfont {
		\centering
		\includegraphics[width=0.45\textwidth]{img/adaptadores.jpg}
		\caption{Orico BTA-406 a esquerda e EDUP N8508GS a direita. Elaborado pelo autor.}
		\label{fig:adaptadores}
	}
\end{figure}

Todo o sistema operacional fica instalado em um cartão microSD, com no mínimo 4 GB de espaço. O cartão utilizado nesse projeto é um SanDisk Ultra Class 10 de 8 GB, conforme figura~\ref{fig:cartaomicrosd}.

\begin{figure}[h!]
	\ABNTEXchapterfont {
		\centering
		\includegraphics[width=0.4\textwidth]{img/cartaomicrosd.jpg}
		\caption{Cartão microSD SanDisk Ultra Class 10 utilizado no projeto. Elaborado pelo autor.}
		\label{fig:cartaomicrosd}
	}
\end{figure}

O sistema Raspbian permite que conecte remotamente via \textit{SSH} (\textit{Secure Shell}), sem necessidade de estar conectado a monitor e teclado. O computador utilizado para realização do projeto é um MacBook Pro, com sistema Mac OS X 10.10. Nesse sistema o uso de \textit{SSH} é simples, bastando abrir o aplicativo Terminal e utilizar o comando "\textit{ssh usuario@computador}", conforme imagem~\ref{fig:ex-terminal}. Esse tipo de abordagem é bastante utilizado para conexão a servidores na nuvem, para executar comandos, softwares, entre outros.

\begin{figure}[h!]
	\ABNTEXchapterfont {
		\centering
		\includegraphics[width=0.8\textwidth]{img/terminal-pi.png}
		\caption{Conexão com o \textit{RPi} via \textit{SSH}. Elaborado pelo autor.}
		\label{fig:ex-terminal}
	}
\end{figure}

O \textit{beacon} utilizado para testes foi o \textit{Zebra MPact}, conforme figura~\ref{fig:zebra-mpact}. Utiliza uma bateria CR2450 para alimentação de energia. A \textit{Zebra} tem um sistema de administração e gerenciamento nomeado \textit{MPact Toolbox}, instalado em um servidor do LTIA com sistema Debian 8.1. Esse sistema foi utilizado somente para conhecimento da tecnologia \textit{beacon} e atualização do \textit{firmware}, feito somente por essa \textit{Toolbox}. Esse software também apresenta a porcentagem de bateria, conforme imagem~\ref{fig:exemplo-toolbox}, permitindo a configuração de vários \textit{beacons} simultaneamente. Permite também a mudança do modo de funcionamento, de \textit{iBeacon} para \textit{MPact}, protocolo criado pela fabricante.

\begin{figure}[h!]
	\ABNTEXchapterfont {
		\centering
		\includegraphics[width=0.6\textwidth]{img/beacon-mpact2.jpg}
		\caption{\textit{Beacon} Zebra MPact utilizado para testes. Elaborado pelo autor.}
		\label{fig:zebra-mpact}
	}
\end{figure}

\begin{figure}[h!]
	\ABNTEXchapterfont {
		\centering
		\includegraphics[width=0.5\textwidth]{img/toolbox-exemplo.png}
		\caption{\textit{Beacon} configurado na \textit{Toolbox} apresentando a porcentagem de bateria. Elaborado pelo autor.}
		\label{fig:exemplo-toolbox}
	}
\end{figure}

Foi utilizado o smartphone Moto Maxx com sistema Android 5.0.1 e o tablet iPad mini Retina com sistema iOS 8.4 confirme imagem~\ref{fig:motomaxx-ipad}. O aplicativo utilizado em ambos foi o \textit{Locate Beacon} da \textit{Radius Networks}, que nos permite identificar os \textit{beacons} e também simular um, para realizar os testes com diferentes tipos, podendo alterar os valores de UUID, Major, Minor e potência de transmissão.

\begin{figure}[h!]
	\ABNTEXchapterfont {
		\centering
		\includegraphics[width=0.6\textwidth]{img/motomaxx-ipad.jpg}
		\caption{Moto Maxx (esquerda) e iPad Mini (direita). Elaborado pelo autor.}
		\label{fig:motomaxx-ipad}
	}
\end{figure}

% ----------------------------------------------------------


% ----------------------------------------------------------
% Experimentos e Resultados
% ----------------------------------------------------------
\chapter{Experimentos e Resultados}
% ----------------------------------------------------------

Os experimentos realizados inicialmente foram a tentativa de identificação de um \textit{beacon} com o \textit{RPi}, para o estudo de funcionamento e comportamento dessas tecnologias. Para isso, o ambiente foi configurado conforme figura~\ref{fig:inicio-ambiente}. 

\begin{figure}[h!]
	\ABNTEXchapterfont {
		\centering
		\includegraphics[width=0.5\textwidth]{img/ambiente1.jpg}
		\caption{Primeiro teste realizado, com \textit{RPi} e \textit{beacon MPact}. Elaborado pelo autor.}
		\label{fig:inicio-ambiente}
	}
\end{figure}

O computador ficou conectado via SSH com o \textit{RPi}, recebendo as informações de leitura de pacotes \textit{BLE}. Os softwares utilizados para isso foram, conforme \citeonline{stack-overflow-ibeacon}, os seguintes:

\begin{alineas}
	\item \textbf{hcitool}: configurado da maneira \textit{hcitool lescan --duplicates}, faz um scan na frequência \textit{BLE} procurando por dispositivos que estejam transmitindo, conforme figura~\ref{fig:hcitool-lescan}.
	\item \textbf{hcidump}: em conjunto com o \textit{hcitool}, apresenta todos os pacotes escaneados na rede. Executado da maneira \textit{hcidump --raw}, conforme figura~\ref{fig:hcidump}.
\end{alineas}

\begin{figure}[h!]
	\ABNTEXchapterfont {
		\centering
		\includegraphics[width=0.8\textwidth]{img/hcitool-lescan.png}
		\caption{Software \textit{hcitool} executando. Elaborado pelo autor.}
		\label{fig:hcitool-lescan}
	}
\end{figure}

\begin{figure}[h!]
	\ABNTEXchapterfont {
		\centering
		\includegraphics[width=0.8\textwidth]{img/hcidump.png}
		\caption{Software \textit{hcidump} executando. Elaborado pelo autor.}
		\label{fig:hcidump}
	}
\end{figure}

Em seguida, com os softwares em execução e os pacotes sendo analisados, o \textit{beacon} foi movido em cima da mesa para ficar a diferentes distâncias do \textit{RPi}, conforme figura~\ref{fig:movimenta-beacon}. Esse passo foi necessário para verificar o formato dos pacotes recebidos e também analisar a distância máxima de identificação.

\begin{figure}[h!]
	\ABNTEXchapterfont {
		\centering
		\includegraphics[width=0.5\textwidth]{img/ambiente4.jpg}
		\caption{Teste com movimentação do \textit{beacon}. Elaborado pelo autor}
		\label{fig:movimenta-beacon}
	}
\end{figure}

Com o adaptador Orico BTA-406 e posicionamento na mesa conforme figuras~\ref{fig:inicio-ambiente} e~\ref{fig:movimenta-beacon}, cerca de 1,3 a 1,4 metros de distância entre o \textit{RPi} e o \textit{beacon} os pacotes já começaram a falhar e a leitura não ser tão constante.

O posicionamento do \textit{RPi} foi alterado para testes, conforme figura~\ref{fig:posiciona-rpi}. Com essa nova posição foi notado uma melhoria na recepção dos pacotes. A 1,5 metros entre o \textit{RPi} e o \textit{beacon} os pacotes começaram a falhar, com recepção notada até 1,6 metros.

\begin{figure}[h!]
	\ABNTEXchapterfont {
		\centering
		\includegraphics[width=0.6\textwidth]{img/ambiente2.jpg}
		\caption{\textit{RPi} posicionado de outra maneira. Elaborado pelo autor.}
		\label{fig:posiciona-rpi}
	}
\end{figure}

Mais experimentos e testes serão feitos durante o andamento do projeto para uma melhor especificação do alcance do \textit{RPi} e \textit{beacon}. Desta forma melhores resultados surgirão.

% ----------------------------------------------------------
% Cronograma
% ----------------------------------------------------------
\chapter{Cronograma}
% ----------------------------------------------------------

Até o momento foram realizadas as atividades de Pesquisa Bibliográfica e Estudo das Tecnologias completamente. Estão em andamento as atividades de Análise e Planejamento e Definição da Estrutura. Serão realizadas as atividades de Implementação, Testes e Finalização do Projeto, conforme cronograma apresentado na Tabela~\ref{table:cronograma}.

\begin{table}[htb]
\IBGEtab{%
\ABNTEXchapterfont {
  \caption{Cronograma de Atividades}%
  \label{table:cronograma}
}
}{%
  \begin{tabular}{ccccccccc}
  \toprule
   Atividade & Jun & Jul & Ago & Set & Out & Nov & Dez & Jan \\
  \midrule \midrule
   Pesquisa Bibliográfica & X & X & X &   &   &   &   &   \\
  \midrule 
   Estudo das Tecnologias &  & X & X &   &   &   &   &   \\
  \midrule 
   Análise e Planejamento &   &   & X & X &   &   &   &   \\
  \midrule 
   Definição da Estrutura &   &   & X & X &   &   &   &   \\
  \midrule 
   Implementação &   &   &   & X & X & X &   &   \\
  \midrule 
   Testes &   &   &   &   & X & X & X &   \\
  \midrule 
   Finalização do Projeto &   &   &   &   &   &   & X & X \\
  \bottomrule
\end{tabular}%
}{%
  \fonte{Elaborado pelo autor.}%
  }
\end{table}

% ----------------------------------------------------------


% ----------------------------------------------------------
% Conclusão
% ----------------------------------------------------------
\chapter{Conclusão}
% ----------------------------------------------------------

Esse projeto tem como objetivo o aprofundamento na área de Internet das Coisas, pelo estudo das tecnologias de \textit{Bluetooth Low Energy}, \textit{Raspberry Pi} e \textit{beacons}. O resultado final será um protótipo de rastreador de \textit{beacons} utilizando o \textit{Raspberry Pi}, ainda em desenvolvimento.

A fundamentação teórica realizada por meio de pesquisa bibliográfica no início do projeto foi essencial para entender como os protocolos \textit{beacons} foram propostas e implementadas, como é o comportamento do \textit{Raspberry Pi}, entre outros. Em conjunto, o estudo das tecnologias foi de extrema importância para entender o funcionamento e utilização do \textit{RPi} e \textit{beacon}.

Certas atividades de extrema importancia ainda serão realizadas, mas o projeto ainda tem muitos frutos para render, caminhando em uma boa direção. A área de Internet das Coisas é uma área extremamente promissora, com áreas de pesquisa e implementação bem vastas. 

% ----------------------------------------------------------


% ----------------------------------------------------------------------------------------------------------------------------------






% ----------------------------------------------------------
% ELEMENTOS PÓS-TEXTUAIS
% ----------------------------------------------------------

% ----------------------------------------------------------

% ----------------------------------------------------------
% Referências bibliográficas
% ----------------------------------------------------------
\bibliography{referencias}

}
\end{document}
% ----------------------------------------------------------
% Introdução
% ----------------------------------------------------------
\chapter{Introdução}\label{cap:introducao}
%\addcontentsline{toc}{chapter}{Introdução}
% ----------------------------------------------------------

A internet cresce dia após dia e cada vez mais diferentes tipos de dispositivos são conectados a essa imensa rede. Há uma estimativa de 26 bilhões de ''coisas'' conectadas a internet até 2020, comparado a 6 bilhões na década de 2000. Isso foi possível graças ao custo do acesso a internet e banda larga ter diminuído 40 vezes nos últimos 10 anos. Esse \textit{boom} de crescimento também é influenciado pela IoT (\textit{Internet of Things} - Internet das Coisas). \cite{goldmansachs-iot}. 

Segundo \citeonline{ashton-iot} o termo IoT surgiu como título de sua apresentação a Procter \& Gamble (P\&G) em 1999. Esse termo se dá ao uso de internet em diferentes tipos de dispositivos. 

\begin{citacao}
''(...) a Internet das Coisas é um conceito no qual dispositivos de nosso dia a dia são equipados com sensores capazes de captar aspectos do mundo real, como por exemplo, temperatura, umidade, presença, etc, e envia-los a centrais que recebem estas informações e as utilizam de forma inteligente.'' \cite{nascimento-iot}.
\end{citacao}

Como exemplo podemos citar: geladeira ligada a internet informando a falta de condimentos, caixa de remédio conectada a internet prevendo o término da caixa de remédio para avisar ao consumidor, entre diversos outros. Um bom exemplo é interligar uma casa por meio de sensores, como termostatos, sistemas de segurança, iluminação, sistemas de entretenimento com uma inteligência por trás para diversas aplicações. \cite{goldmansachs-iot}

Diversas áreas tiveram o seu crescimento alavancado por conta da IoT. O mais marcante é o ramo do DIY (\textit{Do It Yourself}, ou faça você mesmo), em que pessoas criam ou adaptam coisas para suas necessidades. Isso se dá por conta de ter aparecido no mercado dispositivos e componentes que auxiliam a criação e adaptação de eletrônicos e outros. Um ótimo exemplo é o Arduino, um dispositivo que nos ajuda a criar os projetos de eletrônica que consiste de duas partes: o hardware e o software. Com eles é possível construir praticamente de tudo, desde um LED piscante a um robô que envia um \textit{tweet} quando sua planta está sem água. \cite{ben-arduino}. \cite{sorrel-arduino}.

Além do Arduino existem diversos outros \textit{devices} com funcionalidades similares ou até complementares. Podemos citar o \textit{Raspberry Pi}, um computador do tamanho de um cartão de crédito e de baixo custo. \cite{raspberrypi-rpi}. Por ser um computador é possível executar um sistema operacional (como Linux, \textit{RISC OS}, \textit{Windows 10 for IoT09}). Dependendo do modelo possui portas USB (1 a 4) para conexão com periféricos (como mouse, teclado, adaptador WiFi, Bluetooth), e também porta Ethernet para conexão a internet cabeada (exceto modelo A e A+). Também possui de 26 (modelo A e B) a 40 pinos (modelo A+, B+ e 2) para conexões gerais de entrada e saída digitais (GPIO - \textit{General Input and Output}). 

Através desses pinos pode-se conectar uma diversidade de componentes eletrônicos como sensores, atuadores, outros dispositivos para comunicação. Dessa forma, suas funcionalidades são expandidas de uma forma absurda, ficando a cargo de cada pessoa montar uma nova aplicação. Uma boa aplicação é a conexão de um adaptador bluetooth pela USB para comunicação com smartphones, tablets, PCs e outros dispositivos tipo sensores sem fio.

Um exemplo desses sensores externos são os \textit{beacons}, pequenos modelos sem fio que se comunicam por meio do Bluetooth 4.0 ou BLE (\textit{Bluetooth Low Energy} - Bluetooth de Baixa Energia). Como o próprio nome diz, esse padrão de comunicação utiliza pouca energia. Desta forma, um \textit{beacon} pode funcionar por anos. ''Na prática, ela permite localizar objetos (ou pessoas que carregam esses objetos) com muito mais precisão dentro de ambientes fechados.'' \cite{teixeira-beacon}.

Os \textit{beacons} foram pouco explorados até o momento, seu uso está sendo mais notado na área de grandes lojas do varejo.

\begin{citacao}
''A Apple (...) já está utilizando a tecnologia em 254 lojas nos EUA. As funcionalidades já estão embutidas na versão oficial do aplicativo da Apple Store para iOS [, sistema operacional de seus smartphones e tablets]. (...) Quando o usuário se aproxima de uma loja física, o aplicativo oferece toda uma camada extra de informações e serviços que são específicos para aquela unidade – como por exemplo ofertas locais, tamanho da fila para ser atendido no Genius Bar, eventos e treinamentos que estão agendados ali na loja etc.'' \cite{teixeira-beacon}.
\end{citacao}

A Macy's (grande rede norte americana de loja de departamentos) está realizando testes em algumas de suas lojas para enviar alertas a pessoas que entrarem em suas lojas, com promoções e melhores indicações, utilizando o padrão \textit{iBeacon} da Apple. Até o momento esse teste está limitado a usuários de iPhone, e somente quando entrar na loja. Em um teste futuro espera-se que seja possível separar por departamentos, para que quando um usuário percorra a loja apareça as notícias relativo ao local da loja que ela está. \cite{kastrenakes-macys-beacon}

% ----------------------------------------------------------